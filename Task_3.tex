% Beispiel eines Kapitels für die Latex-Vorlage am ETI.
%--------------------------------------------------------------------
%
%   Version: 3.25        (Koma-Skript, documentclass{scrbook})
%   Datum: 01.06.2018
%   Autor: Johannes Bier
%
%--------------------------------------------------------------------

\chapter{Task 3: Investigation of the No-Load Behavior of the Machine}

A no-load measurement is typically performed with the machine terminals open. By rotating the rotor, voltages are induced which can be measured between the individual conductors. These conductor voltages depend on the rotor position angle and exhibit a periodic behavior determined by the machine’s design. 

In a well-designed machine, efforts are made to achieve a voltage waveform that is as sinusoidal as possible. This can be accomplished, for example, by increasing the slot number \gls{sym:q}. In order to achieve an ideal sinusoidal waveform, conductors would need to be distributed homogeneously over the stator. However, since the conductors lie within slots, the sinusoidal shape can only be approximated. Consequently, in addition to the fundamental wave, there are always more or less pronounced harmonic components in the induced voltages.

The following tasks focus on analyzing these effects in more detail.

\bigskip

Tasks:
\begin{enumerate}
	\item Present the waveforms of the induced voltage at \SI{5000}{\per\minute}.
	\item Calculate the induced voltages for \SI{500}{\per\minute}, \SI{1000}{\per\minute}, \SI{5000}{\per\minute}, corner point speed, \SI{10000}{\per\minute} and \SI{15000}{\per\minute}. Plot the max amplitude values versus speed.
	\item Is there a relationship in no-load conditions between the maximum amplitude, the induced voltage, and the motor speed? If so, which physical quantity can be derived from this relationship? Calculate this physical quantity if applicable. Verify your analytical result with the simulation data.
\end{enumerate}

\newpage

\section{Waveforms of the induced voltage at \SI{5000}{\per\minute}}

\section{Speed/induced voltage relation}

\section{Relationship between no-load induced voltage and  motor speed}

\newpage