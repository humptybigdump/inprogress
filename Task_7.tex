% Beispiel eines Kapitels für die Latex-Vorlage am ETI.
%--------------------------------------------------------------------
%
%   Version: 3.25        (Koma-Skript, documentclass{scrbook})
%   Datum: 01.06.2018
%   Autor: Johannes Bier
%
%--------------------------------------------------------------------

\chapter*{Simulation-Based Analysis of a Permanent Magnet Synchronous Machine with Surface Magnets (SPM)}

\chapter{Task 7: Reducing the Cogging Torque}

Minimizing cogging torque is a key concern, especially in automotive applications. Cogging torque can excite vibrations in downstream drivetrain components, potentially causing annoying noise and, over time, fatigue-related wear. Therefore, it is essential to keep this phenomenon as low as possible. The resulting improvement in the drivetrain’s \gls{NVH} characteristics contributes to enhanced vehicle comfort and increased system longevity.

There are various methods to reduce the cogging torque of the motor. The goal of the following task is to investigate and compare these methods. However, it should be noted that these measures can also affect the magnitude of the usable torque of the machine. Hence, a compromise between achievable average torque and torque ripple usually has to be found. This should also be considered in this task.

\bigskip

Tasks:
\begin{enumerate}
	\item Reduction of cogging torque by skew angle: Vary the skew angle in the simulation. Evaluate the corresponding cogging torque as well as the delivered torque.
	\item Rounding of magnets: Vary the rounding of the magnets in the simulation. Evaluate the corresponding cogging torque and the delivered torque.
	\item Fork-shaped tooth tips: Vary the width of the fork shapes in the simulation. Evaluate the corresponding cogging torque and the delivered torque.
	\item Qualitatively evaluate these three measures to reduce cogging torque as well as their impact on the average torque.
	\item [Bonus.] During the design phase, usually more than one parameter is considered, and typically more than one output quantity is evaluated (in this task only the average torque and the cogging torque are evaluated). You may also try combined parameter variations (e.g., combining skew angle and magnet offset) to find an optimal solution.
\end{enumerate}

\newpage

\section{Skew angle as a means of reducing cogging torque}

\section{Magnet rounding as a means of reducing cogging torque}

\section{Fork-shaped tooth tips as a means of reducing cogging torque}

\section{Qualitative comparison of torque-reducing features}

\section*{7.B Combination solution}

\newpage