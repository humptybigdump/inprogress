%%%%%%%%%%%%%%%%%%%%%%%%%%%%%%%%%%%%%%%%%%%%%%%%%%%%%%%%%%%%
\setnextsection{0}
\section{Organisatorisches}
\label{sec}



%%%%%%%%%%%%%%%%%%%%%%%%%%%%%%%%%%%%%%%%%%%%%%%%%%%%%%%%%%%%
\subsection{Bemerkung zu Covid-19}



\begin{frame}[c]{Information zum WS 2020/21}
	\begin{itemize}
		\item Im Hinblick auf die aktuelle Situation bzgl. Covid-19 beachten Sie bitte die folgenden Hinweise für das WS 2020/21:
		
		\begin{itemize}
			\item Die Vorlesung \emph{Wahrscheinlichkeitstheorie} wird im WS 2020/21 \emph{online} stattfinden.
			
			\item Bei Interesse treten Sie bitte dem ILIAS-Kurs bei. Alle Unterlagen und Informationen werden dort gesammelt und verteilt.
			
			\bigskip
			
			\item Die \emph{Vorlesung sowie die Übung} werden zu den regulären\footnote{\scriptsize \schlagwort{Bitte beachten:} Die \emph{regulären Vorlesungszeiten} beginnen aufgrund der durch Covid-19 verursachten Rahmenbedingungen um 8:00, 10:00, 12:00, 14:00, 16:00, 18:00 Uhr!} Vorlesungszeiten als \emph{Videostream mit Zoom} angeboten, wobei Folien mit begleitenden Erklärungen und Notizen erarbeitet werden. Die Videos werden\footnote{\scriptsize ... nach einer gewissen Verzögerung...} in ILIAS bereitgestellt.


			\item Evtl. notwendige Fragestunden oder Zusatzveranstaltungen werden ebenfalls online abgehalten. 

		\end{itemize}
	\end{itemize}
\end{frame}





\begin{frame}[c]{Information zum WS 2020/21}
	\begin{itemize}
		\item Im Hinblick auf die aktuelle Situation bzgl. Covid-19 beachten Sie bitte die folgenden Hinweise für das WS 2020/21:
		
		\begin{itemize}
			
			\item Falls sich die Vorgaben oder der Ablauf im Laufe des Semester ändern, werden die Details in ILIAS angekündigt.
			
			\item Falls Sie irgendwelche Fragen haben, können Sie mich gerne kontaktieren.
			
			\bigskip
			
			\item Leider ist die Situation alles andere als optimal, da Vorlesungen mit persönlichem Kontakt deutlich besser Emotionen und Motivation transportieren sowie Rückfragen und Rückkopplung erlauben. 

			
		\end{itemize}
	\end{itemize}

			
	\eqbox{
		\schlagwort{Mit gemeinsamer Anstrengung, Engagement und "`etwas mehr Forum + Fragestunden"' bekommen wir das gut hin!}
	}
\end{frame}





%%%%%%%%%%%%%%%%%%%%%%%%%%%%%%%%%%%%%%%%%%%%%%%%%%%%%%%%%%%%
\subsection{Literatur}


%%%%%%%
\begin{frame}[allowframebreaks, noframenumbering]{Literatur}
	\begin{thebibliography}{6}
		\bibitem[JW02]{Jondral02}
		F. Jondral, A. Wiesler, \emph{Wahrscheinlichkeitsrechnung und stochastische Prozesse}, Teubner, 2002

		\bibitem[Kre91]{Krengel91}
		U. Krengel, \emph{Wahrscheinlichkeitstheorie und Statistik}, Vieweg, 1991

		\bibitem[Ren62]{Renyi62}
		A. Rényi, \emph{Wahrscheinlichkeitsrechnung -- mit einem Anhang über Informationstheorie}, VEB Verlag der Wissenschaften, 1962

		\bibitem[Hen13]{Henze13}
		N. Henze, \emph{Stochastik für Einsteiger -- Eine Einführung in die faszinierende Welt des Zufalls}, Springer, 2013

		\bibitem[Pap91]{Papoulis91}
		A. Papoulis, \emph{Probability, Random Variables, and Stochastic Processes}, 3rd Ed., New York 1991: McGraw-Hill, Inc.

		\bibitem[Fi77]{Fischer77}
		Prof. A. G. Fischer, \emph{Über das Lernen}, aus dem Vorwort zum Vorlesungsscript "Werkstoffe der Elektrotechnik", Uni Dortmund, 1977

	\end{thebibliography}
\end{frame}
