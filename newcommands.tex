% Definition einiger Befehle/Abkürzungen zur Latex-Vorlage am ETI.
%--------------------------------------------------------------------
%
%   Version: 3.25        (Koma-Skript, documentclass{scrbook})
%   Datum: 01.06.2018
%   Autor: Johannes Bier
%
%--------------------------------------------------------------------

% "def" überm Gleichheitszeichen
\newcommand {\eqdef}{\mathrel{\overset{\makebox[0pt]{\mbox{\normalfont\tiny\sffamily def}}}{=}}}

% Ausrufezeichen überm Gleichheitszeichen      
\newcommand {\eqexcl}{\mathrel{\overset{\makebox[0pt]{\mbox{\normalfont\small\sffamily !}}}{=}}}

% Ausrufezeichen überm "Größer-Gleich"-Zeichen   
\newcommand {\geqexcl}{\mathrel{\overset{\makebox[0pt]{\mbox{\normalfont\small\sffamily !}}}{\geq}}}

% Ausrufezeichen überm "Größer-Gleich"-Zeichen   
\newcommand {\leexcl}{\mathrel{\overset{\makebox[0pt]{\mbox{\normalfont\small\sffamily !}}}{<}}}

% j
\newcommand{\ju}{{\mathrm{j}\mkern1mu}}
% c phi
\newcommand{\cphi}{c \Phi}

% \abs und \norm
\DeclarePairedDelimiter\abs{\lvert}{\rvert}
\DeclarePairedDelimiter\norm{\lVert}{\rVert}

% \Re und \Im
\let\Re\undefined
\let\Im\undefined
\DeclarePairedDelimiter\braces{\lbrace}{\rbrace}
\newcommand{\Re}[1]{\operatorname{Re}\braces*{#1}}
\newcommand{\Im}[1]{\operatorname{Im}\braces*{#1}}

% Trennen von unbekannten Wörtern
\hyphenation{Lja-pu-now}