%%%%%%%%%%%%%%%%%%%%%%%%%%%%%%%%%%%%%%%%%%%%%%%%%%%%%%%%%%%%%%%%%%%%%%%%%%%%%%%%
%2345678901234567890123456789012345678901234567890123456789012345678901234567890
%        1         2         3         4         5         6         7         8

%\documentclass[letterpaper, 10 pt, conference]{orbieeeconfpre}  % Comment this line out if you need a4paper

\documentclass[a4paper, 10pt, journal]{KIT-HCR-Report-IEEE}      % Use this line for a4 paper
%\conference{IEEE Conference for Awesome ORB Research}

\newcommand{\lecturename}{Praktikum: Mathematical and Computational Methods in Robotics \& AI, SS 2025}

% \bibliographystyle{biorob-num}
\bibliographystyle{orbref-num}

\IEEEoverridecommandlockouts                              % This command is only needed if 
                                                          % you want to use the \thanks command

\overrideIEEEmargins                                      % Needed to meet printer requirements.

% See the \addtolength command later in the file to balance the column lengths
% on the last page of the document

\usepackage{hyperref}
\usepackage{graphicx}
\usepackage{tabularx}
\usepackage{booktabs}
\usepackage{lipsum}
\usepackage{amsmath}

\title{\LARGE \bf
Final report  \LaTeX\ template document class*
}

\author{Guybrush U. Threepwood$^{1}$ and Great Student $^{2}$% <-this % stops a space
\thanks{*Based on the guidelines published on the \href{http://conf.papercept.net/conferences/support/tex.php}{PaperCept conference manuscript management website}}% <-this % stops a space
\thanks{$^{1}$Guybrush U. Threepwood is with the Institute for Pirate Sciences, Three-headed Monkey Group, University of  M\^el\'ee Island
        {\tt\small gthreepwood@har.har-har.mi}}%
\thanks{$^{2}$Great student is with Karlsruhe Institute of Technology}%
}


\begin{document}



\maketitle

%%%%%%%%%%%%%%%%%%%%%%%%%%%%%%%%%%%%%%%%%%%%%%%%%%%%%%%%%%%%%%%%%%%%%%%%%%%%%%%%
\begin{abstract}

This demo file is intended to illustrate the \LaTeX\ document class to be used for final report paper in the (pro)seminar based on the style of IEEE publications. 

\end{abstract}

%%%%%%%%%%%%%%%%%%%%%%%%%%%%%%%%%%%%%%%%%%%%%%%%%%%%%%%%%%%%%%%%%%%%%%%%%%%%%%%%
\section{Introduction}
{\bf Important!} Please note that the \verb!KIT-MT-Praktikum-IEEE.cls! file is to be used within the context of the {\it Praktikum: Movement \& Technology} at KIT only and must not be distributed externally!

%% Figure
\begin{figure}[h]
   \centering
   \includegraphics[width=0.1\textwidth]{fig/knubbi.png}
   \caption{Example picture - You can use a reasonable number of figures as needed -  there is no exact limitl}
   \label{fig:knubbi}
\end{figure}

\subsection{Bibliography styles}
The \texttt{orbref-num.bst} bibliography style numbers the citations by their order of appearance. Here are two sample references: \cite{Newton1687,Mombaur2009}.

\subsection{Lorem Ipsum}
\lipsum[1]


\section{Section}
\subsection{Subsection}
This subsection has two subsubsections.

\subsubsection{Subsubsection 1}

\textbf{Bold text} and \textit{Italicized text}

\subsubsection{Subsubsection 2}
% This line should not appear in PDF
\textbf{\textit{Bold and italicized text}} can happen at the same time. The order of bold and italicization doesn't matter. In this sentence, I want to include mathematical expressions: \(\lambda + \gamma - \pi*\alpha*\beta = 42 \pm 0.5 \).

%%%%%%%%%%%%%%%%%%%%%%%%%%%%%%%%%%%%%%%%%%%%%%%%%%%%%%%%%%%%%%%%%%%%%%%%%%%%%%%%
\section{Example section}
\lipsum[1]

% A simpler equation
\begin{equation}
    y = mx + b
    \label{eqn:linear}
\end{equation}

% A slightly more complicated equation
% Need to add \usepackage{amsmath} at the start of this document to use "subequations"
\begin{subequations}
    \begin{align}
        \min_{x(\cdot),u(\cdot)} \quad & \sum_{i=1}^{2} \left( ||q(t_i) - q_{ref}(t_i)||^2 + \int_{{t}_{i-1}}^{{t}_{i}} {\phi(\cdot)dt} \right) \label{ma_ocp_objfunc} \\
        \textrm{s.t.} \quad & \dot{x}(t) = f_{i}(t, x(t), u(t), p) \quad \textrm{for } t \in T \label{ma_ocp_eom}\\
         & r_{eq}(x(0), ..., x(T), p) = 0 \label{ma_ocp_req}\\
         & r_{ineq}(x(0), ..., x(T), p) \geq 0 \label{ma_ocp_rineq}\\
         & g_{i}(t, x(t), u(t), p) \geq 0 \quad \textrm{for } t \in T \label{ma_ocp_g}\\
         & T = [t_1, t_2]^T \label{T}
    \end{align}
    \label{eqn:ma_ocp}
\end{subequations}

\subsection{Example subsection}
\lipsum[2]

%% Table
\begin{table}[h]
\caption{Example table.}
   \begin{tabularx}{0.48\textwidth}{llr}
   		 \toprule 
   		 Symbol & \multicolumn{1}{X}{Description} & Value [m] \\
		 \midrule
		  $A_x$ & Horizontal coordinate of A$^{*}$ & 0.0745 \\
		  $A_z$ & Vertical coordinate of A$^{*}$ & 0.2650 \\  \noalign{\smallskip}
		  
		  $C_x$ & Horizontal coordinate of C$^{\#}$ & 0.0700 \\
		  $C_z$ & Vertical coordinate of C$^{\#}$ & 0.1000 \\
		  $D_x$ & Horizontal coordinate of D$^{\dag}$ & 0.0602 \\
		  $D_z$ & Vertical coordinate of D$^{\dag}$ & 0.0860 \\
		 \bottomrule
   \end{tabularx}  \label{tab:initmodel}
\end{table}



%%%%%%%%%%%%%%%%%%%%%%%%%%%%%%%%%%%%%%%%%%%%%%%%%%%%%%%%%%%%%%%%%%%%%%%%%%%%%%%%
\section{Example section title}
\lipsum[1]

\begin{enumerate}
    \item This is the first numbered line with enumerate
    \item This is the second numbered line with enumerate
\end{enumerate}

\begin{itemize}
    \item This is the first line with itemize
    \item This is the second line with itemize
\end{itemize}

%%%%%%%%%%%%%%%%%%%%%%%%%%%%%%%%%%%%%%%%%%%%%%%%%%%%%%%%%%%%%%%%%%%%%%%%%%%%%%%%
\section{Another example for a section title}

I would like to test this citation to see if the year shows up in the References section \cite{test}.

\lipsum[2]





%%%%%%%%%%%%%%%%%%%%%%%%%%%%%%%%%%%%%%%%%%%%%%%%%%%%%%%%%%%%%%%%%%%%%%%%%%%%%%%%
\section{Conclusions and Outlook}

\lipsum[2]

% \bibliography{./mybibfile}
\bibliography{mybibfile}

\end{document}
